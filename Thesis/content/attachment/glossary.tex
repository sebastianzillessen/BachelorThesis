%%% GLOSSAR %%%

\newglossaryentry{Radiance Map}{
	name=Radiance Map,
	description={Intensität des einfallenden Lichtes in den Aufnahmepunkten einer Szene}
}
\newglossaryentry{Robustheit}{
	name={Robustheit},
	description={Unter Robustheit wird hier der Einsatz von subquadratischen Bestrafungsfunktionen bei der Berechnung verstanden} 
}

\newglossaryentry{Tone-Mapping}{
	name={Tone-Mapping},
	description={Tone-Mapping (dt. Dynamikkompression) steht für die Kompression des Dynamikumfanges von HDR-Bildern auf einen niederen Kontrastbereich, sodass diese mit herkömmlichen Geräten dargestellt werden können}
}

\newglossaryentry{Pentadiagonal-Matrix}{
	name={Pentadiagonal-Matrix},
	description={Quadratische Matrix, bei der nur die fünf zentralen Diagonalen besetzt sind (analog zu Tridiagonal-Matrix)}
}


\newglossaryentry{Dynamikumfang}{
	name={Dynamik\-umfang},
	description={Verhältnis von größter und kleinster Leuchtdichte einer Aufnahme}
}

\newglossaryentry{SaltAndPepperNoise}{
    name={Salt \& Pepper Rauschen},
    description={Das sog. Salt \& Pepper Rauschen beschreibt eine Verkörnung des Bildes, bei Pixel auf weiß oder schwarz gesetzt sind. Dies kann durch Sensor- oder Messfehler entstehen oder durch Übertragungs- oder Abspeicherungsfehler. Salt \& Pepper Rauschen kann künstlich sehr leicht produziert werden, indem einzelne Bildpunkte zufällig auf weiß oder schwarz gesetzt werden }
}

%%% ABKÜRZUNGEN %%%

\newacronym{HDR}{HDR}{High Dynamic Range}
\newacronym{LDR}{LDR}{Low Dynamic Range}
\newacronym{RAW}{RAW}{Rohdatenformate von Kameraaufnahmen}
\newacronym{MTB}{MTB}{Mean Threshold Bitmap Alignment}
\newacronym{CMOS}{CMOS}{Complementary Metal Oxide Semiconductor}
\newacronym{SVD}{SVD}{singular value decomposition (dt. Singulärwertzerlegung)}
\newacronym{SOR}{SOR}{Successive Over-Relaxation (dt. Überrelaxationsverfahren)}
\newacronym{GUI}{GUI}{grafische Benutzerschnittstelle (engl. graphical user interface)}
\newacronym{JRE}{JRE}{Java Runtime Environment}
\newacronym{LGS}{LGS}{Lineare Gleichungssystem}
\newacronym{MVC}{MVC}{Model-View-Controller}
\newacronym{OOP}{OOP}{Objektorientierte Programmierung}




