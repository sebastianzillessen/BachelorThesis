%%% GLOSSAR %%%

\newglossaryentry{Radiance Map}{
	name=Radiance Map,
	description={Test test}
}
\newglossaryentry{Antwortkurve}{
	name={Kamera Antwortkurve},
	description={Test test}
}
\newglossaryentry{Monotonie}{
	name={Monotonie},
	description={Test test}
}
\newglossaryentry{Robustheit}{
	name={Robustheit},
	description={Test test}
}

\newglossaryentry{raeumlicher Glattheitsterm}{
	name={räumlicher Glattheitsterm	},
	description={Test test}
}

\newglossaryentry{Tone-Mapping}{
	name={Tone Mapping},
	description={Test test}
}

\newglossaryentry{RGB-Farbraum}{
	name={RGB-Farbraum},
	description={test}
}

\newglossaryentry{Pentadiagonal-Matrix}{
	name={Pentadiagonal-Matrix},
	description={ist eine (analog zu einer Tridiagonal-Matrix) eine quadratische Matrix, bei der nur die fünf zentralen Diagonalen besetzt sind.}
}

\newglossaryentry{Belichtungswert}{
	name={Belichtungswert},
	description={---},
	plural={Belichtungswerte}
}

\newglossaryentry{Belichtungszeit}{
	name={Belichtungszeit},
	description={---},
	plural={Belichtungszeiten}
}


\newglossaryentry{Dynamikumfang}{
	name={Dynamikumfang},
	description={Verhältnis von größter und kleinster Leuchtdichte.}
}

\newglossaryentry{SaltAndPepperNoise}{
    name={Salt \& Pepper Rauschen},
    description={Das sog. Salt \& Pepper Rauschen beschreibt eine Verkörnung des Bildes, bei Pixel auf weiß oder schwarz gesetzt sind. Dies kann durch Sensor- oder Messfehler entstehen oder durch Übertragungs- oder Abspeicherungsfehler. Salt \& Pepper Rauschen kann künstlich sehr leicht produziert werden, indem einzelne Bildpunkte zufällig auf weiß oder schwarz gesetzt werden. }
}

%%% ABKÜRZUNGEN %%%

\newacronym{HDR}{HDR}{High Dynamic Range}
\newacronym{LDR}{LDR}{Low Dynamic Range}
\newacronym{RAW}{RAW}{}
\newacronym{MTB}{MTB}{Mean Threshold Bitmap Alignment}
\newacronym{CMOS}{CMOS}{Complementary Metal Oxide Semiconductor}
\newacronym{SVD}{SVD}{singular value decomposition (dt. Singulärwertzerlegung)}
\newacronym{SOR}{SOR}{Successive Over-Relaxation (dt. Überrelaxationsverfahren)}
\newacronym{GUI}{GUI}{grafische Benutzerschnittstelle (engl. graphical user interface)}
\newacronym{JRE}{JRE}{Java Runtime Environment}
\newacronym{LGS}{LGS}{Lineares Gleichungssystem}
\newacronym{MVC}{MVC}{Model-View-Controller}




