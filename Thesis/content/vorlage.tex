%Die Angabe des schlauen Spruchs auf diesem Wege funtioniert nur,
%wenn keine Änderung des Kapitels mittels den in preambel/chapterheads.tex
%vorgeschlagenen Möglichkeiten durchgeführt wurde.
\setchapterpreamble[u]{%
\dictum[Albert Einstein]{Probleme kann man niemals mit derselben Denkweise lösen, durch die sie entstanden sind.}
}
\chapter{Kapitel zwei}
\label{chap:k2}
%\vspace{-3cm}
%\vspace{2cm}

Hier wird der Hauptteil stehen. Falls mehrere Kapitel gewünscht, entweder mehrmals \texttt{\textbackslash{}chapter} benutzen oder pro Kapitel eine eigene Datei anlegen und \texttt{ausarbeitung.tex} anpassen.

\section{File-Encoding}
Die Vorlage wurde 2010 auf UTF-8 umgestellt. TeXnicCenter 1 RC 1 unterstützt \textbf{kein} UTF-8. Die Alpha-Version soll UTF-8 können, aber es gibt anscheinend Probleme. Deshalb bitte einen anderen Editor, wie \zB TeXstudio.

\section{Zitate}
Wörter am besten mittels \texttt{\textbackslash enquote\{...\}} \enquote{einschließen}, dann werden die richtigen Anführungszeichen verwendet.

\section{Mathematische Formeln}
\label{sec:mf}
Mathematische Formeln kann man $so$ setzen. \texttt{symbols-a4.pdf} (zu finden auf \url{http://www.ctan.org/tex-archive/info/symbols/comprehensive/symbols-a4.pdf}) enthält eine Liste der unter LaTeX direkt verfügbaren Symbole. z.\,B.\ $\mathbb{N}$ für die Menge der natürlichen Zahlen. Für eine vollständige Dokumentation für mathematischen Formelsatz sollte die Dokumentation zu \texttt{amsmath}, \url{ftp://ftp.ams.org/pub/tex/doc/amsmath/} gelesen werden.

Folgende Gleichung erhält keine Nummer, da \texttt{\textbackslash equation*} verwendet wurde.
\begin{equation*}
x = y
\end{equation*}

Die Gleichung~\autoref{eq:test} erhält eine Nummer:
\begin{equation}
\label{eq:test}
x = y
\end{equation}

Eine ausführliche Anleitung zum Mathematikmodus von LaTeX findet sich in \url{http://www.ctan.org/tex-archive/help/Catalogue/entries/voss-mathmode.html}.

\section{Quellcode}
Listing~\autoref{lst:ListingANDlstlisting} zeigt, wie man Programmlistings einbindet.  Mittels \texttt{\textbackslash lstinputlisting} kann man den Inhalt direkt aus Dateien lesen.

%Listing-Umgebung wurde durch \newfloat{Listing} definiert
\begin{Listing}
\begin{lstlisting}
<listing name="second sample">
  <content>not interesting</content>
</listing>
\end{lstlisting}
\caption{lstlisting in einer Listings-Umgebung, damit das Listing durch Balken abgetrennt ist}
\label{lst:ListingANDlstlisting}
\end{Listing}

Quellcode im \lstinline|<listing />| ist auch möglich.

\section{Abbildungen}
Die Abbildungen~\autoref{fig:chor1} und~\autoref{fig:chor2} sind für das Verständnis dieses Dokuments
wichtig. Im Anhang zeigt Abbildung~\autoref{fig:AnhangsChor} erneut die komplette Choreographie.

%Die Parameter in eckigen Klammern sind optionale Parameter - z.B. [htb!]
%htb! bedeutet: "Liebes LaTeX, bitte platziere diese Abbildung zuerst hier ("_h_ere"). Falls das nicht funktioniert, dann bitte oben auf der Seite ("_t_op"). Und falls das nicht geht, bitte unten auf der Seite ("_b_ottom"). Und bitte, bitte bevorzuge hier und oben, auch wenn's net so optimal aussieht ("!")
%Diese sollten nach Möglichkeit NICHT verwendet werden. LaTeX's Algorithmus für das Platzieren der Gleitumgebung ist schon sehr gut!
%\begin{figure}
%  \begin{center}
%    \includegraphics[width=\textwidth]%{choreography.pdf}
%    \caption{Beispiel-Choreographie}
%    \label{fig:chor1}
%  \end{center}
%\end{figure}

%\begin{figure}
%  \begin{center}
%    \includegraphics[width=.8\textwidth]%{choreography.pdf}%
%    \caption[Beispiel-Choreographie]{Die Beispiel-%Choreographie. Nun etwas kleiner, damit %\texttt{\textbackslash textwidth} demonstriert wird. Und auch die Verwendung von alternativen Bildunterschriften für das Verzeichnis der Abbildungen. Letzteres ist allerdings nur Bedingt zu empfehlen, denn wer liesst schon so viel Text unter einem Bild? Oder ist es einfach nur Stilsache?}
%    \label{fig:chor2}
%  \end{center}
%\end{figure}

Das SVG in \autoref{fig:directSVG} ist direkt eingebunden, während der Text im SVG in \autoref{fig:latexSVG} mittels pdflatex gesetzt ist. Falls man die Graphiken sehen möchte, muss inkscape im PATH sein und im Tex-Quelltext \verb1\iffalse1 und \verb1\iftrue1 auskommentiert sein.

\iffalse
\begin{figure}
\centering
\includegraphics{svgexample.svg}
\caption{SVG direkt eingebunden}
\label{fig:directSVG}
\end{figure}

\begin{figure}
\centering
\def\svgwidth{.4\textwidth}
\includesvg{svgexample}
\caption{Text im SVG mittels \LaTeX{} gesetzt}
\label{fig:latexSVG}
\end{figure}
\fi

\section{Tabellen}
Tabelle~\autoref{tab:Ergebnisse} zeigt Ergebnisse.
\begin{table}
  \begin{center}
    \begin{tabular}{ccc}
	\toprule
	\multicolumn{2}{c}{\textbf{zusammengefasst}} & \textbf{Titel} \\ \midrule
	Tabelle & wie & in \\
	\url{tabsatz.pdf}& empfohlen & gesetzt\\
	
	\multirow{2}{*}{Beispiel} & \multicolumn{2}{c}{ein schönes Beispiel}\\
	 & \multicolumn{2}{c}{für die Verwendung von \enquote{multirow}}\\
	\bottomrule
    \end{tabular}
    \caption[Beispieltabelle]{Beispieltabelle -- siehe \url{http://www.ctan.org/tex-archive/info/german/tabsatz/}}
    \label{tab:Ergebnisse}
  \end{center}
\end{table}

\section{Pseudocode}
Algorithmus~\autoref{alg:sample} zeigt einen Beispielalgorithmus.
\begin{Algorithmus} %Die Umgebung nur benutzen, wenn man den Algorithmus ähnlich wie Graphiken von TeX platzieren lassen möchte
\caption{Sample algorithm}
\label{alg:sample}
\begin{algorithmic}
\Procedure{Sample}{$a$,$v_e$}
\State $\mathsf{parentHandled} \gets (a = \mathsf{process}) \lor \mathsf{visited}(a'), (a',c,a) \in \mathsf{HR}$ 
\State \Comment $(a',c'a) \in \mathsf{HR}$ denotes that $a'$ is the parent of $a$
\If{$\mathsf{parentHandled}\,\land(\mathcal{L}_\mathit{in}(a)=\emptyset\,\lor\,\forall l \in \mathcal{L}_\mathit{in}(a): \mathsf{visited}(l))$}
\State $\mathsf{visited}(a) \gets \text{true}$
\State $\mathsf{writes}_\circ(a,v_e) \gets
\begin{cases}
\mathsf{joinLinks}(a,v_e) & \abs{\mathcal{L}_\mathit{in}(a)} > 0\\
\mathsf{writes}_\circ(p,v_e)
& \exists p: (p,c,a) \in \mathsf{HR}\\
(\emptyset, \emptyset, \emptyset, false) & \text{otherwise}
\end{cases}
$
\If{$a\in\mathcal{A}_\mathit{basic}$}
  \State \Call{HandleBasicActivity}{$a$,$v_e$}
\ElsIf{$a\in\mathcal{A}_\mathit{flow}$}
  \State \Call{HandleFlow}{$a$,$v_e$}
\ElsIf{$a = \mathsf{process}$} \Comment Directly handle the contained activity
  \State \Call{HandleActivity}{$a'$,$v_e$}, $(a,\bot,a') \in \mathsf{HR}$
  \State $\mathsf{writes}_\bullet(a) \gets \mathsf{writes}_\bullet(a')$
\EndIf
\ForAll{$l \in \mathcal{L}_\mathit{out}(a)$}
  \State \Call{HandleLink}{$l$,$v_e$}
\EndFor
\EndIf
\EndProcedure
\end{algorithmic}
\end{Algorithmus}

\clearpage
Und wer einen Algorithmus schreiben möchte, der über mehrere Seiten geht, der kann das nur mit folgendem \textbf{üblen} Hack tun:

{
\begin{minipage}{\textwidth}
\hrule height .8pt width\textwidth
\vskip.15em%\vskip\abovecaptionskip\relax
\stepcounter{Algorithmus}
\addcontentsline{alg}{Algorithmus}{\protect\numberline{\theAlgorithmus}{\ignorespaces Description \relax}}
\noindent\textbf{Algorithmus \theAlgorithmus} Description
%\stepcounter{algorithm}
%\addcontentsline{alg}{Algorithmus}{\thealgorithm{}\hskip0em Description}
%\textbf{Algorithmus \thealgorithm} Description
\vskip.3em%\vskip\belowcaptionskip\relax
\hrule height .5pt width\textwidth
\end{minipage}
code goes here\\
test2\\
\vskip-.9em
\hrule height .5pt width\textwidth
}

\section{Verweise}
Verweise auf einen Abschnitt gehen mittels: ``Siehe Abschnitt~\autoref{sec:mf}''. Das Kommando \texttt{\textbackslash{}vref} funktioniert ähnlich wie \texttt{\textbackslash{}ref} mit dem Unterschied, dass zusätzlich ein Verweis auf die Seite hinzugefügt wird. \texttt{vref}: Abschnitt \autoref{sec:diff}, \texttt{ref}: \autoref{sec:diff}.

%Mit MiKTeX Installation ab dem 2012-01-16 nicht mehr nötig
%Falls ein Abschnitt länger als eine Seite wird und man mittels \texttt{\textbackslash{}vref} auf eine konkrete Stelle in der Section
%verweisen möchte, dann sollte man \texttt{\textbackslash{}phantomsection} verwenden und dann wird
%auch bei \texttt{vref} die richtige Seite angeben.

%%The link location will be placed on the line below.
%%Tipp von http://en.wikibooks.org/wiki/LaTeX/Labels_and_Cross-referencing#The_hyperref_package_and_.5Cphantomsection
%\phantomsection
%\label{alabel}
%Das Beispiel für \texttt{\textbackslash{}phantomsection} bitte im \LaTeX{}-Quellcode anschauen.

%Hier das Beispiel: Siehe Abschnitt \vautoref{hack1} und Abschnitt \vautoref{hack2}.

\section{Verschiedenes}
\label{sec:diff}
\ifdeutsch
Ziffern (123\,654\,789) werden schön gesetzt. Falls dies nicht gewünscht ist, den Parameter \texttt{osf} bei dem Paket \texttt{mathpazo} herausnehmen.
\fi

\textsc{Kapitälchen} werden schön gesperrt...

\begin{compactenum}[I]
\item Man kann auch die Nummerierung dank paralist kompakt halten
\item und auf eine andere Nummerierung umstellen
\end{compactenum}

\section{Weitere Informationen}
Verbesserungsvorschläge für diese Vorlage sind immer willkommen. Bitte bei github ein Ticket eingragen (\url{https://github.com/latextemplates/USTUTT-computer-science/issues}).
