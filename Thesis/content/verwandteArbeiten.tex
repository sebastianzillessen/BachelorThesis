%\setchapterpreamble[u]{%
%\dictum[Albert Einstein]{Probleme kann man niemals mit derselben Denkweise lösen, durch die sie entstanden sind.}
%}
\chapter{Verwandte Arbeiten und Implementierungen}
\label{chap:references}

In diesem Kapitel soll kurz auf verwandte Arbeiten und Implementierungen eingegangen werden. Der Fokus liegt bei der nachfolgenden Recherche auf dem Ansatz aus einer Belichtungsserie von \gls{LDR}-Bildern ein \gls{HDR}-Bild zu generieren.

\section{Bekannte Implementierungen des Ansatzes von Debevec und Malik}
\label{sec:implementations}
Das Standard-Verfahren als solches wurde bereits in verschiedenen Programmen implementiert. Am ursprünglichen Artikel von Debevec und Malik ist bereits eine \texttt{MATLAB}-Version des Algorithmus angefügt. 
Darauf basierend hat z.B. Mathias Eitz eine Implementierung\footnote{\url{http://cybertron.cg.tu-berlin.de/eitz/hdr/index.html}} des kompletten Prozesses in \texttt{MATLAB} geschrieben. Dieser arbeitet ohne Erweiterungen und implementiert direkt den beschriebenen Ansatz. Die \nameref{algo:schwachstellen:selektion} (siehe \autoref{algo:schwachstellen:selektion}) geschieht in dieser Implementierung in einer Art Rasterung der Bilder und berücksichtigt keine der Forderungen von Debevec und Malik (siehe \autoref{algo:schwachstellen:selektion}). 

Auch in der beschriebenen Software zur Erstellung von \gls{HDR}-Bildern (siehe \autoref{sec:software}) wird z.T. dieser Ansatz verwendet.


\section{Verwandte Arbeiten}

In den letzten Jahren haben die veröffentlichten Arbeiten zur Generierung, Darstellung und Verarbeitung von \gls{HDR}-Bildern stetig zugenommen. Deswegen hat die nachfolgende Auflistung keinen Anspruch auf Vollständigkeit und dient lediglich einer groben Übersicht.

Nayar et al. \cite{Nayar00highdynamic} stellen in ihrem Verfahren einen anderen Ansatz der Generierung von \gls{HDR}-Bildern vor. Dabei wird bereits bei der Aufnahme eines Bildes eine Rasterung durch ein optisches Gitter mit unterschiedlichen Transparenzen erzielt. Das so aufgenommene Bild wird als spatially varying exposure (dt. ortsabhängig belichtetes) Bild bezeichnet. Da die Struktur des Gitters und dessen Transparenzen bekannt sind, kann aus dem aufgenommenen Bild nun ein \gls{HDR}-Bild mit höherem Dynamikumfang berechnet werden. Die unterschiedlichen Transparenzen des Gitters sorgen dafür, dass sowohl hohe als auch niedrige Belichtungen wahrgenommen werden können. 

Jinno and Okkuda \cite{Jinno} beschreiben in ihrer Alternative für die Fusion von Belichtungsserien (basierend auf herkömmlichen Algorithmen) auch die Problematik von sich bewegenden Objekten. Daraus entstehen bei der Fusion häufig ghosting artifacts (dt. Geist-Artefakte) oder motion blur (dt. Bewegungsunschärfe). In dieser Veröffentlichung werden die bewegten Objekte erkannt und bei der Berechnung des \gls{HDR}-Bildes wieder entfernt. Das Verfahren sagt dabei Überdeckung, Sättigung und Verschiebungen in den Ausgangsbildern voraus und konstruiert die \gls{HDR}-Bilder dann unter Berücksichtigung dieser Daten. Damit können insbesondere in Serien mit hoher Bewegung sehr viel bessere Ergebnisse erzielt werden.

Auch die \gls{Tone-Mapping}-Operatoren werden ständig untersucht und verbessert. So vergleichen Kuang et al. \cite{tone_mapper_2} in ihrer Studie über \gls{HDR}-Bildgenerierungs-Algorithmen vier lokale und zwei globale Operatoren miteinander. Während der Studie werden verschiedene Bildszenen mit den sechs Operatoren von Probanden in drei unterschiedlichen Experimenten bewertet. Dazu werden die Bilder paarweise auf einem \gls{LDR}-Bildschirm gezeigt. Das Experiment stellte fest, dass keiner der verwendeten Operatoren durchweg in allen Szenen besser abgeschnitten hat als die Mitgetesteten. Dies führt zur Schlussfolgerung, dass eine große Korrelation zwischen Szene und \gls{Tone-Mapping}-Verfahren vorliegt. 

In einer Arbeit von Yoshida et al. \cite{tone_mapper_1} werden sieben \gls{Tone-Mapping}-Operatoren im Direktvergleich der realen Szene und dem korrespondierendem \gls{LDR}-Bild von Testpersonen bewertet. Berücksichtigt wurden dabei sowohl globale als auch lokale Operatoren. Eine der Haupterkenntnisse dieser Studie war, dass die lokalen Operatoren die Bilddetails besser beibehalten und die globalen Operatoren den Kontrast besser darstellen können.

Liu et al. \cite{Xinqiao} beschreiben in ihrem Artikel ein heuristisches Verfahren zur Schätzung der Bewegung in einer Bildserie. Dieses Verfahren basiert auf einem in selbigem Artikel veröffentlichten rekursiven Verfahren bei dem große Belichtungsserien (ihr Beispiel umfasst 65 Aufnahmen) Stück für Stück zu einem \gls{HDR}-Bild zusammengesetzt werden. Ihr Ansatz verspricht besonders bei Bildern mit sehr schnellen Änderungen (wie z.B. einem sich drehenden Propeller) gute Ergebnisse. Die Aufnahmen selbst werden dabei durch herkömmliche \gls{CMOS}-Bildsensoren aufgenommen. Mögliche Messfehler werden im Algorithmus ausgiebig behandelt um Rauschen zu reduzieren.

Im Artikel von Zimmer et al. \cite{zimmer} wird ebenfalls ein Verfahren zur Reduktion von Bewegungsunschärfe bei der Erzeugung von \gls{HDR}-Bildern beschrieben. Hierbei kommt die Berechnung des optischen Flusses bei der Registrierung der Bilder zum Einsatz. Darüber hinaus wird in diesem Verfahren ein hochauflösendes \gls{HDR}-Bild erzeugt, da aus den verschiedenen Bildern der Belichtungsserie durch die Registrierung auch Zwischenpixel-Bereiche mit Informationen gefüllt werden können. Dadurch kann die Auflösung erhöht werden. Mithilfe dieses Verfahrens ist es möglich auch verwackelte Bilder, die Bewegung enthalten, zu registrieren und dadurch ein \gls{HDR}-Bild zu erzeugen.