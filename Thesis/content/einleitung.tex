\chapter{Einleitung}
Die \gls{HDR}-Bildgebung ist eines von vielen interessanten Problemen in dem aufstrebenden Forschungsgebiet \textit{Computational Photography}. Ziel dieser Arbeit ist die Fusion mehrerer Bilder mit verschiedener Belichtungszeit zu einem einzigen Bild mit deutlich vergrößertem Dynamikumfang.
 
\section{Motivation}


Während viele Arbeiten sich nur mit der pixelweisen Fusion der Bilddaten auseinander setzen, schlagen Debevec und Malik \cite{paper} vor, gleichzeitig auch noch die Antwortkurve des Bildaufnahmeprozesses, d.h. der verwendeten Kamera, mitzuschätzen (siehe \autoref{fig:antwortkurve}). Dies bietet den klaren Vorteil, die Bildfusion auch ohne vorherige radiometrische Kalibration des Aufnahmeequipments durchführen zu können. Als mathematisches Werkzeug zur Formulierung des Verfahrens dient hierbei ein gemeinsames Energiefunktional, das einen Ähnlichkeits- und einen  Glattheitsterm besitzt. Während der Ähnlichkeitsterm unter Berücksichtigung der mitgeschätzten Antwortkurve die Beziehung zwischen den Einzelaufnahmen und dem gesuchten HDR-Bild herstellt, sorgt der Glattheitsterm für eine hinreichend glatte Antwortkurve, die auch aus radiometrischer Sicht Sinn ergibt.

\begin{figure}
  \begin{center}
    \includegraphics[width=\textwidth]{ImageAquisitionPipeline}
    \caption{\textit{Bildaufnahme Pipeline} --- Veranschaulichung der zu durchlaufenden Prozesse bei der Aufnahme eines Bildes mit einer Kamera  \cite[S.2]{paper}.}
    \label{fig:antwortkurve}
  \end{center}
\end{figure}

Trotz der allgemeinen Formulierung hat das Verfahren von Debevec und Malik jedoch auch einige Schwachstellen. Zum einen werden weder im Daten- noch im Glattheitsterm robuste Bestrafungsfunktionen verwendet. Diese könnten den Ansatz deutlich robuster unter Fehlmessungen machen. Zum anderen werden keine Beschränkungen gefordert, die die typischerweise gewünschte Monotonie der Antwortkuve explizit erzwingen würden. Monotone Kurven können deshalb nur bei einer hinreichend großen Gewichtung der Glattheit erzielt werden. Schließlich ist das Verfahren auch nicht sonderlich robust gegenüber Rauschen, was insbesondere bei sehr kurz belichteten Bildern Probleme bereiten kann. 

\section{Aufgabenstellung}
Ziel der Arbeit ist es zunächst das Verfahren von Debevec und Malik \cite{paper} als Ausgangsverfahren in einer (dafür sinnvollen und portierbaren) Programmiersprache zu implementieren.

Diese Implementierung soll anschließend sukzessive um eine Monotonie-Beschränkung (siehe \autoref{sec:monotonie}), einen räumlichen Glattheitsterm (siehe \autoref{sec:raeumlich}) und robuste Bestrafungsfunktionen (siehe \autoref{sec:robustheit}) erweitert werden.

\begin{figure}[H]
  \begin{center}
      \begin{overpic}[width=0.48\textwidth]{teezer/E_noise_no_robustness}
                \put(-0,0){\includegraphics[width=2cm]{teezer/g_noise_no_robustness}}
        \end{overpic}
        \hfill
        \begin{overpic}[width=0.48\textwidth]{teezer/E_noise_robustness_raum}
            \put(-0,0){\includegraphics[width=2cm]{teezer/g_noise_robustness_raum}}
        \end{overpic}
    \caption{\textit{Antwortkurven (incl. zugehörigem HDR-Bild)} --- Das HDR-Bild wurde mittels lokalem Reinhard-Tone-Mapper (siehe \autoref{subsec:ToneMapping}) erstellt. \textbf{links}: Ohne robuste Bestrafungsterme. \textbf{rechts}: Mit robusten Bestrafungstermen und räumlicher Glattheitsforderung.}
    \label{fig:teezer}
  \end{center}
\end{figure}


Neben der Modellierung und Implementierung der einzelnen Erweiterungen soll auch eine geeignete visuelle Evaluation der Ergebnisse erfolgen. Hierzu sollen \gls{Tone-Mapping}-Verfahren (siehe \autoref{subsec:ToneMapping}) aus bereits existierender Forschung verwendet werden. 


\section{Gliederung}
Die Arbeit gliedert sich wie folgt:
\begin{description}

\item[\autoref{chap:hdr} -- \nameref{chap:hdr}:] Hier werden die Grundlagen der \gls{HDR}-Bilder vermittelt. Dabei wird auf die physikalischen, historischen und anwendungsorientierten Eigenschaften von \gls{HDR}-Bildern eingegangen.

\item[\autoref{chap:references} -- \nameref{chap:references}:] Anschließend werden verwandte Arbeiten zum Thema \gls{HDR} vorgestellt.

\item[\autoref{chap:algo} -- \nameref{chap:algo}:] Die zu Grunde liegende Vorgehensweise von Debevec und Malik soll in diesem Kapitel beschrieben werden. Außerdem werden die existierenden Schwachstellen des bisherigen Ansatzes dargestellt.

\item[\autoref{chap:maths} -- \nameref{chap:maths}:] Der theoretische Ansatz aus \autoref{chap:algo} wird hier mathematisch umgesetzt und diskutiert. Außerdem werden die Erweiterungen des Algorithmus beschrieben und formal spezifiziert.

\item[\autoref{chap:impl} -- \nameref{chap:impl}:] Dieses Kapitel beschäftigt sich mit der Herangehensweise an die Problemstellung aus Sicht der Entwicklung. Es werden Anforderungen an die Software formuliert, der Prototyp vorgestellt und die Architektur der Implementierung beschrieben. Abschließend wird die Verwendung der Software erläutert.

\item[\autoref{chap:results} -- \nameref{chap:results}:] Anschließend werden die Resultate und Einflüsse der unterschiedlichen Erweiterungen vorgestellt und diskutiert.

\item[\autoref{chap:zusfas} -- \nameref{chap:zusfas}:] Zusammenfassung der Ergebnisse der Arbeit und Darstellung von Anknüpfungspunkten zu weiteren Arbeiten.
\end{description}
